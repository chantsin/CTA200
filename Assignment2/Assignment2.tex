\documentclass{article}
\usepackage[utf8]{inputenc}
\usepackage{natbib}
\usepackage{graphicx}
\usepackage[margin=1in]{geometry}
\usepackage{amsfonts}
\usepackage{amsmath}

\title{CTA200 Assignment 2}
\author{Calvin Chan}
\date{May 2020}


\begin{document}
\maketitle
\section{Mandelbrot set}
\subsection{Methods}
Using numpy, we apply the restrictions given to us, where $-2 < x < 2$, $-2 < y < 2$, and $z_0 = 0$. We also apply the equation to be iterated using a for-in statement while setting a boundary for the absolute value of z with the letter alpha. This allows us to distinguish between values that will diverge versus values that will converge. We then plot the graph using Matplotlib as shown below. 
\subsection{Analysis}
When graphed we can see that this iteration resembles the famous Mandelbrot set. We can see that it has a symmetrical feature along the y axis. When zoomed in, it also shows similar repeated patterns emerging, which in fact goes on infinitely.
\begin{figure}[h!]
    \centering
    \includegraphics{Mandelbrot BW.png}
    \caption{Mandelbrot set with varying values of c}
    \label{fig:Mandelbrot_set_BW}
\end{figure}

\newpage{}
We also implemented colours which allows use to distinguish the near cutoff points between convergence and divergence.

\begin{figure}[h!]
    \centering
    \includegraphics{Mandelbrot Colour.png}
    \caption{Mandelbrot set with colour bar}
    \label{fig:Mandelbrot_set_colour}

    \centering
    \includegraphics{Mandelbrot Colour Zoom.png}
    \caption{Zoomed in Mandelbrot set with colour bar}
    \label{fig:Mandelbrot_set_colour_zoom}
\end{figure}

\newpage{}

\section{The SIR Model}
\subsection{Methods}
We started off by importing scipy.integrate's odeint in order to integrate as well as defining the parameters given with set values or values of our choice. Then we implemented the derivatives of our functions that are to be integrated. After that, we expressed those results in the form a graph over the time interval from 0 to 200. Two additional graphs were made with varying beta and gamma values in order to analyse the behaviour of the function. 
\subsection{Analysis}
It turns out that depending on the infection rate and recovery rate (or beta and gamma), the rate at which people are either susceptible, infected, or recovered can vary drastically. Higher values of beta and gamma indicate that the spread is much quicker but it also settles down faster. In other words, if it is more contagious, then the faster it will spread but also the faster the curve will flatten out and regress. 

\begin{figure}[h!]
    \centering
    \includegraphics[scale=0.75]{SIR Model 1.png}
    \caption{SIR Model 1, $\beta=0.2$, $\gamma=1./10$}
    \label{fig:SIR_Model_1}
    
    \centering
    \includegraphics[scale=0.75]{SIR Model 2.png}
    \caption{SIR Model 2, $\beta=0.6$, $\gamma=2./10$}
    \label{fig:SIR_Model_2}
\end{figure}

\newpage

\begin{figure}
    \centering
    \includegraphics[scale=0.75]{SIR Model 3.png}
    \caption{SIR Model 3, $\beta=0.4$, $\gamma=3./10$}
    \label{fig:SIR_Model_3}
\end{figure}

\end{document}
